In this section, the layer is described in terms of the hardware and software design. Specific implementation details, such as hardware components, programming languages, software dependencies, operating systems, etc. should be discussed. Any unnecessary items can be ommitted (for example, a pure software module without any specific hardware should not include a hardware subsection). The organization, titles, and content of the sections below can be modified as necessary for the project.
   Layer Y is mobile application layer. This layer includes mobile app featuring barcode scanner, and internal and private database to keep track of inventory of product. The hardware component in this layer is android phone. We need android phone featuring camera on both sides to scan and read the beer. Generally, we are making our android app in android studio, and we are u. sing java as programming language. The app consists of user interface, internal database zxing library to read the barcode and procced to look into created when app is launched for the first time.

\subsection{Layer Hardware}
Our beer inventory app is mainly focused software rather than hardware. It consists of mobile app featuring barcode scanner connected with internal database to keep track of inventory product. In this layer hardware component is android phone. As to scan the beer and keep track of beer on the basis of their location, type, style and name. All of the codes and programs running in the app act as software.

\subsection{Layer Operating System}
The operating system required by layer is user interface and android os. In this layer, User interact with the mobile app features, its content and function. Similarly, in this user interact with app to scan the item with barcode   placed on the basis of name, type, style and location. 

\subsection{Layer Software Dependencies}
Our app is software based. We are using android studio for making beer app. The programming language we are using is java. Library used in mobile app is zxing library.

\subsection{Subsystem 1}
The subsystem of mobile app layer is barcode scanner. This subsystem scans the barcode number and get the product information like type, price, style and date from the database with respective private and public connectors. Barcode reader is connected with public database connector through barcode number. It performs to get the information of beer based on its types, origin, price, style and date. It extracts the information by its barcode number available in database.

\subsubsection{Subsystem Hardware}
There is no hardware used in this sub system. There is barcode reader that acts with brewing product to get brewing number and all the information.

\subsubsection{Subsystem Operating System}
The operating system act on this subsystem is user interface. User interact with app to scan the item with barcode   placed on the basis of name, type, style and location.

\subsubsection{Subsystem Software Dependencies}
There are many libraries used while making the android app. Library used in mobile app is zxing library.

\subsubsection{Subsystem Programming Languages}
We are using android studio for making beer app. The programming language used in this subsystem is java.

\subsubsection{Subsystem Data Structures}
In this subsystem, when barcode reader scans the beer items; the information/data of beer is stored in internal database. 

\subsubsection{Subsystem Data Processing}
In this subsystem, the algorithm used are insertion, counting sort and search algorithm. When barcode scan the beer, it helps to find the quantity ,  and search om the basis of type , style and name.
\subsection{Subsystem 2}
Another subsystem of mobile app is private database connectors. The app scans a bar-code and proceeds to look for the number in the internal database, if the product has already been registered then the app will present the available information to the user but in case that the product is not found in the database the app will allow the user to add the product to the inventory. In this subsystem, the manually added information is stored in the app database internally. All information is stored in private database as we are adding all the information manually

\subsubsection{Subsystem Hardware}
Our app mainly focused on software. There is no hardware used in this subsystem. However, there is internal database storage when the items are added manually in the beer app. 

\subsubsection{Subsystem Operating System}
The operating system act on this layer is memory management or storage

\subsubsection{Subsystem Software Dependencies}
There are many libraries used while making the android app. Library used in mobile app is zxing library.

\subsubsection{Subsystem Programming Languages}
We are using android studio for making beer app. The programming language used in this subsystem is java.We are using local database(sQlite) to add beer manually and to store the information.

\subsubsection{Subsystem Data Structures}
In this subsystem, all the information and data are stored in local database. They are added manually in the app and after adding it stored internally in app database.

\subsubsection{Subsystem Data Processing}
In this subsystem, the algorithm used are search algorithm. It helps to search the information from the database. It also used insertion algorithm ; adding information manually.

