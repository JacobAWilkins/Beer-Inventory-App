During the developmental stages of Swift Scan, the Android and Desktop apps will be tested for performance to make sure that information storage and retrieval are executed fast when the user interacts with the GUI.  The Android app, which will be used as a search engine for brewery products, will perform quickly to establish connection to the database and retrieve the product information for the beverage choice of the user.  On the other hand, the destop app will allow the user to easily manage the information that is stored in the database.  The goal of the desktop app is to make it feasible for the customer to use and navigate.  Since a barcode scanner will be used for the overall product, thorough testing and development will be done to make sure the barcode scanner does not consume repidly the smartphone's battery.

\subsection{Swift Scan Performance Metrics}
\subsubsection{Description}
For best performance measures, both the Android and desktop app will be developed to not consume much of the battery nor processing time for a smartphones processor.  Both applications will be written in Java and the Android app will use Anroid's API in Android Studio to develop the user interface.  The functionality of both applications will consist of using up-to-date API's for the feel-and-look of the GUI and also any data that must be stored and retrieved from the database will consist of using the MySQL commands within the code for faster performance of both applications. 
\subsubsection{Source}
CSE Senior Design project specifications
\subsubsection{Constraints}
Due to the budget, we won't know how the Android app will respond on different smartphones running Android OS.  Furthermore, the charging port on smartphones is different for different brands and the barcode scanner that will be purchased will be used on Samsung smartphones.
\subsubsection{Standards}
For the performace requirements, no applicable standards will apply in this section.
\subsubsection{Priority}
Moderate
