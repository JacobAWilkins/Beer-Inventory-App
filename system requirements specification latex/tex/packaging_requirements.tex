The Beverage Invetory App (Swift Scan) will be available for download through GitHub for Android.  The GitHub link to download the Android app will be made available once the app has been fully tested and developed.  In addition, a desktop application will also be made available to download from GitHub and will need to be installed on a PC running Windows OS.  The desktop application will be an executable file so it can properly run on Windows OS.  The main purpose of the desktop application is for the user to have administrative privileges and access to adding, deleting, or modifying records from the central database that'll store all the information for any beverage.  Both the Android and desktop applications need to be both installed on an Android smartphone and Windows OS, respectively, in order for Swift Scan to function properly.  The barcode scanner device for the Android app will be packaged in a box with the name and logo of the application, Swift Scan.  An instruction manual will be included with the barcode scanner with information for both the Android and desktop applications.

\subsection{Swift Scan Packaging Requirements}
\subsubsection{Description}
The Android app along with the desktop app will be developed for the user interface to be feasible and have a 'comfortable look' that will enable the user to interact with both applications in a convenient manner.  In order for the user to understand the functionality of both the Android and desktop applications, the user manual will be written with clear understanding on the functionality of every aspect of the applications and also how the information is stored in the database. 
\subsubsection{Source}
UTA's Laser Safety Manual
\subsubsection{Constraints}
For packaging requirements, the ideal constraint that we might run into is the shortage of barcode scanners in the market that could make it difficult to find and purchase a barcode scanner that is convenient for the customer.
\subsubsection{Standards}
Barcode scanner will meet the requirements to scan barcodes that are part of the EAN/UPC standard, which will scan 1D barcodes from the beverage products.
\subsubsection{Priority}
High
