Constraints are limitations imposed on the project, such as the limitation of cost, schedule, or resources, and you have to work within the boundaries restricted by these constraints. All projects have constraints, which are defined and identified at the beginning of the project.

Constraints are outside of your control. They are imposed upon you by your client, organization, government regulations, availability of resources, etc. Occasionally, identified constraints turn out to be false. This is often beneficial to the development team, since it removes items that could potentially affect progress.

This section should contain a list of at least 5 of the most critical constraints related to your project. For example:

The following list contains key constraints related to the implementation and testing of the project.

\begin{itemize}
  \item For the RFID scanner, the total cost should not exceed $1,000.00.
  \item Due to members of the team having part-time or full-time jobs, an equal amount of tasks have been assigned to each members to have         the protoype finished in time with testing done numerous times to make sure the application is communicating efficiently with the         RFID scanner.
  \item Android application along with the functionality and communication to the RFID scanner must be completed by the first week of             December 2019, which would be from December 1st through December 7th.
  \item The Android application will be developed to meet different versions of Android operating systems out in the market.
  \item The RFID scanner must be able to scan 1D barcodes, which are always found on any beverage product and thus wil make it easy for           the user to scan any 1D barcode from a beverage product.
\end{itemize}
