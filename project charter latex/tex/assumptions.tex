An assumption is a belief of what you assume to be true in the future. You make assumptions based on your knowledge, experience or the information available on hand. These are anticipated events or circumstances that are expected to occur during your project's life cycle.

Assumptions are supposed to be true but do not necessarily end up being true. Sometimes they may turn out to be false, which can affect your project significantly. They add risks to the project because they may or may not be true. For example, if you are working on an outdoor unmanned vehicle, are you assuming that testing space will be available when needed? Are you relying on an external team or contractor to provide a certain subsystem on time? If you are working at a customer facility or deploying on their computing infrastructure, are you assuming you will be granted physical access or network credentials?

This section should contain a list of at least 5 of the most critical assumptions related to your project. For example:

The following list contains critical assumptions related to the implementation and testing of the project.

\begin{itemize}
  \item A secure database with a Linux operating system (OS) for storing information on each beverage as part of the inventory.  The most         suitable database would be using UTA's Omega database server.
  \item For secure communication between the application on an Android smartphone and the database server, a protocol will be esbablished         using port 443, or more commonly known as https.
  \item Access to different versions of Android OS to make sure that the functionality of the application will interact and function                correctly with the RFID scanner.
  \item The RFID scanner along with the SDK that comes with it should be able to scan any barcodes from beverages without the difficulty           of the user scanning any beverage barcode.
  \item Android smartphones should be compatible to fit with the RFID scanner of our choice so the smartphone does not fall or come apart         from the RFID scanner.
\end{itemize}
