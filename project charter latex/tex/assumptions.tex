An assumption is a belief of what you assume to be true in the future. You make assumptions based on your knowledge, experience or the information available on hand. These are anticipated events or circumstances that are expected to occur during your project's life cycle.

Assumptions are supposed to be true but do not necessarily end up being true. Sometimes they may turn out to be false, which can affect your project significantly. They add risks to the project because they may or may not be true. For example, if you are working on an outdoor unmanned vehicle, are you assuming that testing space will be available when needed? Are you relying on an external team or contractor to provide a certain subsystem on time? If you are working at a customer facility or deploying on their computing infrastructure, are you assuming you will be granted physical access or network credentials?

This section should contain a list of at least 5 of the most critical assumptions related to your project. For example:

The following list contains critical assumptions related to the implementation and testing of the project.

\begin{itemize}
  \item A suitable outdoor testing location will be available by the 3rd sprint cycle
  \item The X sensing system developed by Sensor Consulting Company will be delivered according to specifications by the 4th sprint cycle
  \item Access to the customer installation site will be provided by the 5th sprint cycle
  \item The customer will provide ample power and network connectivity at the installation site
  \item The installation site network infrastructure will allow TCP network traffic on port 8080
\end{itemize}